% Abstract

\pdfbookmark[1]{Abstract}{Abstract} % Bookmark name visible in a PDF viewer

\begingroup
\let\clearpage\relax
\let\cleardoublepage\relax
\let\cleardoublepage\relax

\chapter*{Resumen} % Abstract name

La Comunidad Web Caché trata de establecer una comunidad en la cual cada miembro comparta un poco de sus recursos computacionales como almacenamiento, procesamiento y ancho de banda para buscar el bien común. Esto se traduce en un incremento en la eficiencia y la velocidad del servidor web que sirve archivos. Los principales protocolos utilizados en el proyecto son HTTP y P2P. Y es con la combinación de ambos, más la especificación de un nuevo protocolo, que se logra el diseño de la CWC. El diseño toma en cuenta mecanismos de consistencia de datos, de distribución equitativa de trabajo a través de la comunidad virtual y de recuperación de fallos, para darle continuidad al negocio. 

\endgroup			

\vfill