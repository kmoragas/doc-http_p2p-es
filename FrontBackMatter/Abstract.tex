% Abstract

\pdfbookmark[1]{Abstract}{Abstract} % Bookmark name visible in a PDF viewer

\begingroup
\let\clearpage\relax
\let\cleardoublepage\relax
\let\cleardoublepage\relax

\chapter*{Resumen} % Abstract name

El objetivo de la Comunidad Web Caché, es establecer una comunidad en la cual cada miembro comparta un poco de sus recursos computacionales, como los son: el almacenamiento, el procesamiento y el ancho de banda, con el propósito de buscar el bien común. Esto se traduce en un incremento en la eficiencia y la velocidad del servidor web que sirve los archivos hacia los clientes. Los principales protocolos utilizados en el proyecto son HTTP y P2P. 

La combinación de éstos dos últimos protocolos y la especificación de un nuevo llamado CWC, son los ingredientes para alcanzar el diseño de la Comunidad Web Caché. 

El diseño toma en cuenta mecanismos de consistencia de datos, distribución equitativa de trabajo a través de la comunidad virtual y recuperación de fallos, con el objetivo de brindar continuidad al negocio, transparencia, decrecimiento de los costos y un aumento en la velocidad de respuesta del sitio web que implemente el CWC. 

\endgroup			

\vfill