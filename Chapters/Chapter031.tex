% Chapter 3

\chapter{Protocolo P2P} % Chapter title

\label{ch:protocolo_p2p} % For referencing the chapter elsewhere, use \autoref{ch:protocolo_p2p}

%----------------------------------------------------------------------------------------

Una red peer-to-peer (P2P) o red de pares, es una red de computadoras en la que todos o algunos aspectos de ésta funcionan sin clientes ni servidores fijos, sino una serie de nodos que se comportan como iguales entre sí. Es decir, actúan simultáneamente como clientes y servidores respecto a los demás nodos de la red.[1]
Las redes peer-to-peer aprovechan, administran y optimizan el uso del ancho de banda de los demás usuarios de la red por medio de la conectividad entre los mismos, obteniendo más rendimiento en las conexiones y transferencias que con algunos métodos centralizados convencionales, donde una cantidad relativamente pequeña de servidores provee el total del ancho de banda y recursos compartidos para un servicio o aplicación.
Alfredo F. nos dice que de forma simple puede verse como la comunicación entre pares o iguales utilizando un sistema de intercambio [6].

Los sistemas compañero a compañero (o P2P por sus siglas en inglés) permiten que computadoras de usuario final se conecten directamente para formar comunidades, cuya finalidad sea el compartir recursos y servicios computacionales. En este modelo, se toma ventaja de recursos existentes en los extremos de la red, tales como tiempo de CPU y espacio de almacenamiento. Las primeras aplicaciones  emergentes se orientaban a compartir archivos y a la mensajería [6].


%----------------------------------------------------------------------------------------

