% Chapter 6

\chapter{Comunidades Virtuales} % Chapter title

\label{ch:comunidades_virtuales} % For referencing the chapter elsewhere, use \autoref{ch:name} 

%----------------------------------------------------------------------------------------

\section{Definición de una Comunidad Virtual}

Las Comunidades Virtuales (CV) son lugares en la web donde las personas se pueden encontrar y "hablar" electrónicamente entre sí. Estas comunidades normalmente están rodeadas por intereses comunes. En un inicio las CV actuaban como si fuera una cafetería en lugar de un asunto meramente comercial, aunque éstas eran por naturaleza comerciales, donde las personas realizaban transacciones y comercio \cite{gupta:2005}. 

Las nociones de "comunidad" y "virtual", plantean la tentativa de definir una comunidad virtual como "una congregación de cibernautas que integran una comunidad que aparenta ser real al simular los efectos de las congregaciones sociales humanas reales o tradicionales, pero sin llenar todas las características de estas".

Por otro lado, se podría definir CV como: Se denomina comunidad virtual a aquella comunidad cuyos vínculos, interacciones y relaciones tienen lugar no en un espacio físico sino en un espacio virtual como Internet.


%------------------------------------------------

\section{Características de una Comunidad Virtual}

Según Whittaker \cite{whittaker:1997} una comunidad debe por lo menos tener los siguiente atributos esenciales: 

\begin{enumerate}
\item Los miembros deben de compartir un mismos objetivo, interés, necesidad o actividad que provea una razón primaria para pertenecer a una comunidad.  
\item Los miembros deben de estar comprometidos a una participación activa y en ocasiones interacciones intensas, donde se compartan actividades y sentimientos comunes entre los participantes.
\item Los miembros deben tener acceso a los recursos compartidos y además deben de existir políticas para determinar el acceso a éstos recursos.
\item Debe existir una reprocidad de la información, soporte y servicios entre los miembros.
\item Compartir un contexto, puede ser social, de lenguaje o simples protocolos. 
\end{enumerate}

Por otro lado algunos atributos menos necesarios podrían ser \cite{whittaker:1997}:
\begin{enumerate}
\item Roles y reputación diferenciada.
\item Conciencia de los límites de los miembros y de la identidad del grupo.
\item Criterio de iniciación.
\item Un historial de participación.
\item Eventos o rituales
\item Un espacio físico compartido
\item Una membresía voluntaria. 
\end{enumerate}

%------------------------------------------------

\section{Tipos de Comunidades Vituales}

A grandes rasgos podemos clasificar las CV en tres grandes categorías: de ocio, profesionales y de aprendizaje \cite{cabero}. 

Por otro lado para Armstrong y Hagel \cite{armstrong:1997} existen dos tipos de Comunidades Virtuales: las orientadas a los usuarios y las orientadas a la organización. 

\begin{description}

\item[Orientadas a los usuarios: ] En las orientadas a los usuarios, son ellos los que definen el tema de la Comunidad y se pueden dividir en:

\begin{itemize}

\item \emph{Geográficas}: agrupan personas que viven o que están interesadas en intercambiar   Información sobre una misma área geográfica.
\item \emph{Temáticas}: orientadas a la discusión de un tema de interés para los usuarios.
\item \emph{Demográficas}: reúnen usuarios de características demográficas similares.
\item \emph{De ocio y entretenimiento}: dirigidas a aquellos cibernautas que ocupan su tiempo libre en juegos en red. Se crean por tipos de juegos como estratégicos, de simulación, etc.
\item \emph{Profesionales}: para aquellos expertos en una materia que desarrollan su actividad concreta en un área profesional definida, generalmente asociada a una formación superior. Especialmente en el caso de las profesiones liberales, cuando se trabaja de manera independiente.
\item \emph{Gubernamentales}: Los organismos gubernamentales han creado Comunidades Virtuales a las que puede acudir el ciudadano para informarse y/o discutir.
\item \emph{Eclécticas}: son aquellas Comunidades Virtuales mixtas, que intentan un poco de todo: zona de ocio, una vía de transmisión y comportamiento cultural, etc.


\end{itemize}

\item [Orientadas a la organización: ]
Por el contrario en las orientadas hacia la organización, el tema es definido según los objetivos y áreas de trabajo de la organización donde reside la comunidad, y podemos dividirlas en:

\begin{itemize}
\item \emph{Verticales}: que aglutinan a usuarios de empresas de diferentes ramas de actividad económica o a organizaciones institucionales.
\item \emph{Funcionales}: referidas a un área específica del funcionamiento de la organización, por ejemplo: mercadeo, producción, relaciones públicas.
\item \emph{Geográficas}: que se concentran en una zona geográfica cubierta por la organización.

\end{itemize}

\end{description}

Por el contrario Salinas \cite{salinas:2003} agrupa las Comunidades Virtuales utilizando un enfoque más pragmático: 

\begin{description}
\item[De discurso: ] 
El ser humano es la criatura social y puede hablar cara a cara sobre intereses comunes, pero también puede compartir estos intereses con otros semejantes más lejanos mediantes los medios de comunicación. 

\item[De práctica: ] 
Cuando en la vida real alguien necesita aprender algo, normalmente no abandona su situación normal y dedica su esfuerzo en clases convencionales sino que puede formar grupos de trabajo (comunidades de práctica), asigna roles, enseña y apoya a otros y desarrolla identidades que son definidas por los roles que desempeña en el apoyo al grupo. 

\item[De construcción de conocimiento: ] 
El objetivo de este tipo de comunidades es apoyar a los estudiantes a perseguir estratégica y activamente el aprendizaje como una meta. 

\item[De aprendizaje: ] 
Si una comunidad es una organización social de personas que comparten conocimiento, valores y metas, las clases como las conocemos no son comunidades ya que los estudiantes están desconectados o están compitiendo unos con otros. Las comunidades de aprendizaje surgen cuando los estudiantes comparten intereses comunes. Las TIC pueden contribuir a conectar alumnos de la misma clase o de alrededor del mundo, con el objeto de lograr objetivos comunes. 

\end{description}

Una última clasificación realizada por Cabero \cite{cabero} apoyado en las conclusiones de Salinas \cite{salinas:2003}. Identifica una serie de grupos en función de: 

\begin{description}
\item[Modo de Asignación: ]
Dado el modo de asignación se pueden encontrar:
\begin{itemize}
\item Comunidades de asignación libre por parte de los miembros.
\item Comunidades de asignación voluntaria.
\item Comunidades de asignación obligatoria.
\end{itemize}
\item[Función primaria de la comunidad: ]
Las comunidades también se pueden clasificar por su función:
\begin{itemize}
\item \emph{Distribución}: Cuando la principal función de la comunidad radica en la distribución de información o mensajes, entre los miembros.
\item \emph{Compartir}: Se trata de comunidades donde prima el intercambio de experiencias y recursos.
\item \emph{Creación}: Cuando se generan procesos de trabajo colaborativo con el objeto de lograr materiales, documentos, proyectos compartidos.
\end{itemize}
\item[Gestión de la comunidad: ]
Por otro lado un punto importante que nos puede ayudar a clasificar una CV es el tipo de gestión que se utilice:
\begin{itemize}
\item \emph{Abiertas}: Cuando el acceso (independientemente de la asignación) es abierto y los recursos de la comunidad están a disposición tanto de los miembros como de personas ajenas a la comunidad.
\item \emph{Cerradas}: Cuando existe algún procedimiento que impide a las personas ajenas a la comunidad el acceso, de tal forma que los recursos, materiales, producciones, histórico, entre otros; sólo son accesibles para los miembros de la comunidad.
\end{itemize}
\item[El objeto de la comunidad: ]
En la línea de la clasificación descrita anteriormente, las comunidades virtuales de aprendizaje podemos clasificarlas en función del objeto que persiguen, en:
\begin{itemize}
\item Comunidades de aprendizaje propiamente dichas, cuando han sido creadas para que el grupo humano que se incorpora a la comunidad desarrolle procesos de aprendizaje en programas diseñados al efecto.
\item Comunidades de práctica, ya definidos anteriormente
\item Comunidades de investigación, cuando se trata de comunidades que desarrollando actividades de aprendizaje, el objeto principal es poner en marcha proyectos de investigación conjunta de acuerdo con la filosofía del trabajo cooperativo a través de redes.
\item Comunidades de innovación. Similares a las anteriores que buscan compartir, intercambiar y generar procesos de innovación en distintos campos.
\end{itemize}
\end{description}

%----------------------------------------------------------------------------------------
