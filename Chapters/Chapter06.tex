% Chapter X

\chapter{Comunidades Virtuales} % Chapter title

\label{ch:comunidades_virtuales} % For referencing the chapter elsewhere, use \autoref{ch:name} 

%----------------------------------------------------------------------------------------

\section{Definición de una Comunidad Virtual}

Bueno ya hemos hablado un poco de historia pero en realidad ¿qué son las comunidades virtuales?
A pesar de que también se les designa como "congregaciones electrónicas" "comunidades en línea" "comunidades electrónicas"; el término más usado es el de comunidad virtual y está compuesto por dos nociones: la de "comunidad" y "virtual". De allí que, en aras de aproximarnos a su definición es necesario analizar por separado el concepto de estos dos términos.
Definiremos primero el término Comunidad.  Etimológicamente, Foster (1997) afirma que el término comunidad tiene un linaje directo con la palabra comunicación y a su vez, Merril y Loewenstein (1979) plantean que este último "proviene del latín communis (común) o communicare (el establecimiento de una comunidad o comunalidad)" (Foster, 1997 pag. 24). En este respecto, el autor advierte que aún cuando la comunicación es la base de la comunidad, ambos términos no deben confundirse, ya que un individuo puede comunicarse con otro sin que formen parte de una misma comunidad.
Por otro lado, tenemos el término virtual. Desde el punto de vista histórico Wilbur (1997) afirma que la palabra virtual data de la edad media y se originó a partir de la palabra "virtud". Durante esta era, se usaba el término virtual para calificar el poder divino, porque tenía la "virtud" de ser real aun cuando no se pudiera observar en el mundo material. Esta es la primera vertiente semántica del término: lo virtual es algo " que tiene virtud para producir un efecto" ( Sopena, Diccionario Enciclopédico Ilustrado 1965, pag. 3697).
Las nociones de "comunidad" y "virtual", plantean la tentativa de definir una comunidad virtual como "una congregación de cibernautas que integran una comunidad que aparenta ser real al simular los efectos de las congregaciones sociales humanas reales o tradicionales, pero sin llenar todas las características de estas".
Por otro lado, también tenemos la siguiente definición: Se denomina comunidad virtual a aquella comunidad cuyos vínculos, interacciones y relaciones tienen lugar no en un espacio físico sino en un espacio virtual como Internet.


%------------------------------------------------

\section{Características de una Comunidad Virtual}

Las Comunidades Virtuales, sólo existen y funcionan en la medida en que sean fruto de la actividad de los ciudadanos, entendidos estos como individuos, colectivos formales o informales, empresas, organizaciones, etc. Así se han creado espacios artificiales (virtuales) nuevos, dotados de una serie de características no siempre comprensibles desde los parámetros del "mundo real". 

\begin{description}
\item[La información es de los usuarios: ]
Es decir, la Red, en principio, está "vacía" y son los usuarios quienes deciden qué información van a almacenar, mostrar e intercambiar. Por tanto, cada usuario decide por dónde empieza a ver la Red, para qué y con quiénes.


\item[Acceso a la red: ] 
Un punto importante que debe de considerar una CV es precisamente, establecer principios de acceso a la red. 

\begin{itemize}
\item \emph{Universal}: basta acceder a un ordenador de la red, para acceder a toda la red o "ver" toda la Red (otra cosa es que, una vez dentro de la Red, haya lugares donde se pida el registro para acceder a la información que contienen)
\item \emph{Simultáneo}: todos estamos en la Red al mismo tiempo, pues existimos en cuanto información (ceros y unos). En realidad, la red es desde sus orígenes el primer contestador automático que se puso en funcionamiento. Nadie sabe si estamos conectados o no, pero nos relacionamos entre todos como si lo estuviéramos a través de nuestra presencia numérica, de la información que "movemos" y de las interacciones que promovemos; Independiente del tiempo (24h./365d.) y de la distancia. Es el primer espacio abierto constantemente a la actividad del ser humano independientemente de donde se encuentre. Sólo necesita acceder a un ordenador de la Red para que todo lo expuesto anteriormente funcione.

\end{itemize}

\item[Organización de la red: ] Finalmente, los otros dos rasgos que cierran este comprimido código genético es que la red crece de manera descentralizada y desjerarquizada. Basta seguir añadiendo ordenadores (servidores) para que se esparza física y virtualmente y no hay ordenadores que desempeñen tareas de "comando y control" sobre los otros ordenadores de la Red.

\end{description}


%------------------------------------------------

\section{Tipos de Comunidades Vituales}

A grandes rasgos podemos clasificar las CV en tres grandes categorías: de ocio, profesionales y de aprendizaje. Aunque algunos autores como Polo (1998) nos indica que pueden darse:
\begin{itemize}
\item Centrada en las personas: Las personas se reúnen fundamentalmente para disfrutar del placer de la mutua compañía.
\item Centrada en el tema: Las personas comparten un tema en común. 
\item Centrada en un acontecimiento: Personas centradas en acontecimientos externos.
\end{itemize}

Para Hagel y Armstrong (1997) hay dos tipos claramente diferenciados, las orientadas hacia el usuario y las orientadas hacia la organización. 

\begin{description}

\item[Orientadas a los usuarios: ] En las orientadas a los usuarios, son ellos los que definen el tema de la Comunidad y se pueden dividir en:

\begin{itemize}

\item \emph{Geográficas}: agrupan personas que viven o que están interesadas en intercambiar   Información sobre una misma área geográfica.
\item \emph{Temáticas}: orientadas a la discusión de un tema de interés para los usuarios.
\item \emph{Demográficas}: reúnen usuarios de características demográficas similares.
\item \emph{De ocio y entretenimiento}: dirigidas a aquellos cibernautas que ocupan su tiempo libre en juegos en red. Se crean por tipos de juegos como estratégicos, de simulación, etc.
\item \emph{Profesionales}: para aquellos expertos en una materia que desarrollan su actividad concreta en un área profesional definida, generalmente asociada a una formación superior. Especialmente en el caso de las profesiones liberales, cuando se trabaja de manera independiente.
\item \emph{Gubernamentales}: Los organismos gubernamentales han creado Comunidades Virtuales a las que puede acudir el ciudadano para informarse y/o discutir.
\item \emph{Eclécticas}: son aquellas Comunidades Virtuales mixtas, que intentan un poco de todo: zona de ocio, una vía de transmisión y comportamiento cultural, etc.


\end{itemize}

\item [Orientadas a la organización: ]
Por el contrario en las orientadas hacia la organización, el tema es definido según los objetivos y áreas de trabajo de la organización donde reside la comunidad, y podemos dividirlas en:

\begin{itemize}
\item \emph{Verticales}: que aglutinan a usuarios de empresas de diferentes ramas de actividad económica o a organizaciones institucionales.
\item \emph{Funcionales}: referidas a un área específica del funcionamiento de la organización, por ejemplo: mercadeo, producción, relaciones públicas.
\item \emph{Geográficas}: que se concentran en una zona geográfica cubierta por la organización.

\end{itemize}

\end{description}

En una línea similar a la anterior, Salinas (2003) nos llama la atención respecto a dentro de las CV se pueden distinguir una serie de grupos en función de:

\begin{description}
\item[Modo de Asignación: ] 
Dado el modo de asignación se pueden encontrar: 

\begin{itemize}
\item Comunidades de asignación libre por parte de los miembros.
\item Comunidades de asignación voluntaria.
\item Comunidades de asignación obligatoria.
\end{itemize}


\item[Función primaria de la comunidad: ] 
Las comunidades también se pueden clasificar por su función:  

\begin{itemize}
\item \emph{Distribución}: Cuando la principal función de la comunidad radica en la distribución de información o mensajes, entre los miembros.
\item \emph{Compartir}: Se trata de comunidades donde prima el intercambio de experiencias y recursos.
\item \emph{Creación}: Cuando se generan procesos de trabajo colaborativo con el objeto de lograr materiales, documentos, proyectos compartidos.
\end{itemize}


\item[Gestión de la comunidad: ] 
Por otro lado un punto importante que nos puede ayudar a clasificar una CV es el tipo de gestión que se utilice: 

\begin{itemize}
\item \emph{Abiertas}: Cuando el acceso (independientemente de la asignación) es abierto y los recursos de la comunidad están a disposición tanto de los miembros como de personas ajenas a la comunidad.
\item \emph{Cerradas}: Cuando existe algún procedimiento que impide a las personas ajenas a la comunidad el acceso, de tal forma que los recursos, materiales, producciones, histórico, etc., sólo son accesibles para los miembros de la comunidad.
\end{itemize}


\item[El objeto de la comunidad: ] 
En la línea de la clasificación descrita anteriormente, las comunidades virtuales de aprendizaje podemos clasificarlas en función del objeto que persiguen, en:

\begin{itemize}
\item Comunidades de aprendizaje propiamente dichas, cuando han sido creadas para que el grupo humano que se incorpora a la comunidad desarrolle procesos de aprendizaje en programas diseñados al efecto.
\item Comunidades de práctica, ya definidos anteriormente
\item Comunidades de investigación, cuando se trata de comunidades que desarrollando actividades de aprendizaje, el objeto principal es poner en marcha proyectos de investigación conjunta de acuerdo con la filosofía del trabajo cooperativo a través de redes.
\item Comunidades de innovación. Similares a las anteriores que buscan compartir, intercambiar y generar procesos de innovación en distintos campos.
\end{itemize}

\end{description}


Abordando más expresamente el último grupo planteado por el profesor Salinas, nos encontramos con la propuesta de Jonassen, Pech, y Wilson (1998) que establecen cuatro tipos de comunidades virtuales:

\begin{description}
\item [De discurso:] El ser humano es una criatura social y puede hablar cara a cara sobre intereses comunes, pero también puede compartir estos intereses con otros semejantes más lejanos mediante los medios de comunicación. Las redes de ordenadores proporcionan numerosas y potentes herramientas para el desarrollo de este tipo de comunidades.

\item [De práctica:] Cuando en la vida real alguien necesita aprender algo, normalmente no abandona su situación normal y dedica su esfuerzo en clases convencionales, sino que puede formar grupos de trabajo (comunidades de práctica), asigna roles, enseña y apoya a otros y desarrolla identidades que son definidas por los roles que desempeña en el apoyo al grupo.

\item [De construcción de conocimiento:] El objetivo de este tipo de comunidades es apoyar a los estudiantes a perseguir estratégica y activamente el aprendizaje como una meta.

\item [De aprendizaje:] Si una comunidad es una organización social de personas que comparten conocimiento, valores y metas, las clases como las conocemos no son comunidades ya que los estudiantes están desconectados o están compitiendo unos con otros. Las clases son comunidades sociales, pero su propósito no es aprender juntos o unos de otros, antes parece que estos grupos buscan reforzar socialmente sus propias identidades por exclusión de los otros.

\end{description}

%----------------------------------------------------------------------------------------
