% Chapter 1

\chapter{Introducción} % Chapter title

\label{ch:introduction} % For referencing the chapter elsewhere, use \autoref{ch:introduction} 

%----------------------------------------------------------------------------------------

Durante los últimos años el incremento en el uso de Internet ha sido exagerado, diariamente aparecen cientos de sitios web ofreciendo información muy variada, el acceso a Internet paso de estar en una cuantas manos a ser usado diariamente por la mayoría de las personas en el mundo. La expansión del Internet se consolido durante el aparecimiento del protocolo HTTP en la década de los noventas, este protocolo vino a establecer una forma universal para intercambiar información, el protocolo se baso en la arquitectura cliente-servidor, en esta arquitectura un cliente realiza la petición de algún recurso al servidor y el servidor satisface esta petición mediante el envió del recurso solicitado.
Debido al auge del protocolo HTTP se dio la expansión del Internet, esta expansión trajo consigo que se creara una gran red de comunicación a nivel mundial, en estos días por ejemplo podemos acceder a información de último minuto en cualquier parte del mundo y la mayor cantidad de internautas, poseen su propio sitio web o alguna página personal hospedada por algún servidor gratuito. Conforme el uso Internet fue creciendo, también la tecnología fue evolucionando, llegando a que actualmente podemos acceder a gran cantidad de información instantáneamente y la mayoría de los usuarios de Internet ignoran todo lo que implica esto, cada vez que acceden a un gigabyte de información situado en el otro lado del mundo no tienen una idea de la tecnología involucrada ni mucho menos la inversión realizada para poder entregar esta información en un tiempo aceptable.
Ahora nos preguntamos ¿qué justifica esta mejora en la tecnología tanto desde el punto de vista de protocolo así como dispositivos involucrados?, la respuesta es muy sencilla, al ser HTTP un protocolo tan versátil, un lenguaje universal para el intercambio de datos, se comenzó a utilizar para diferentes aplicaciones, se paso de un simple intercambio de información, a ser la piedra angular por la cual se mueve la mayoría de las actividades humanas, hoy en día tenemos muchos usos, desde la aparición de la multimedia, los requerimientos de todas las partes involucradas (cliente, servidor y medio de comunicación) se ha incrementado, hoy en día compartimos videos de gran tamaño, se transmite televisión en alta definición, se comparte música, documentos, en fin un gran número de archivos de diferentes tamaños, que hasta hace unos años con nuestra conexión de 56Kbps parecía imposible.
Pero ¿acaso esta maravilla tecnológica es gratuita?, y la respuesta es no, como lo mencionamos anteriormente cada parte involucrada requiere adquirir la nueva tecnología, en el caso de un cliente imaginamos a una persona en su casa la cual paga su conexión a Internet de banda ancha para compartir archivos, revisar su correo electrónico, escuchar música, podemos hablar de ver películas, participar de algún juego en línea, participar en video conferencias; con estos requerimientos el equipo requerido es bastante costoso, pero digamos que no inaccesible, ahora vámonos hacia el otro extremo, el servidor, supongamos un servidor que hace transmisión de videos, los cuales son aportados por los usuarios del sitio que se encuentra hospedado en este servidor, imaginemos que el sitio recibe un gran número de visitas, hablemos de más de 10000 visitas diarias, ya es un numero grande, ahora usted como lector y experto se dirá, resuelvo fácilmente este problema, compro un gran número de servidores, con los cuales utilizando una solución de balanceo de carga, puedo hacer frente a este gran número de usuarios, pensemos por un momento en nuestras palabras tomemos el tiempo e investigamos el precio de un servidor, de una solución de NLB y de una conexión de Internet lo suficientemente buena como para servir videos, ya los números son bastante grandes, lo más preocupante es que nunca es suficiente, siempre tendremos que estar actualizándonos ya que lo único constante es que la tecnología es cambiante.
Debido al problema expuesto anteriormente, aparece como solución el caching, en su definición más sencilla podemos decir que es un modelo mediante el cual, objetos que se encuentran hospedados en un sitio web son almacenados en otros servidores de forma que pueden ser servidos desde otros servidores distintos al servidor web original, esto reduce el número de peticiones que llegan al servidor original lo que disminuye su carga (en la sección 3 de este documento se repasaran los modelos de web caching existentes), el caching ha sido un tema de estudio en cientos de tesis e investigaciones, pero las soluciones usadas actualmente se limitan a soluciones que deben ser adquiridas por el dueño del sitio web y como hablamos no es muy rentable, y soluciones que son adquiridas por un cliente en su beneficio. Al parecer llegamos a un callejón sin salida.
Desviemos nuestra atención de nuestra reflexión hacia dos aspectos de suma importancia que fueron traídos con el auge del Internet, el primero es el establecimiento de comunidades virtuales, en las cuales un grupo de Internautas se reúnen alrededor de un sitio web que presenta información de su interés y como comunidad ayudan para mantenerlo y tienen su propio conjunto de reglas, y el segundo aspecto es el compartir archivos en Internet, tal vez el más polémico de todos y el que más nos interesa es el protocolo P2P, el cual básicamente permite que un número de usuarios que cuentan con un archivo especifico lo compartan con otros de manera simultánea, esto quiere decir que si 100 usuarios tienen un archivo de mi interés y este tiene un tamaño de 100MB, en lugar de bajarlo desde un solo servidor, bajo pequeñas partes desde cada uno de los usuarios que lo tienen, con esto el proceso de descarga es mucho más rápido y no se satura a un solo servidor.
Para ir finalizando esta extensa introducción, el presente documento pretende introducir una posible solución de bajo costo para resolver el problema que se ha expuesto, esta solución la hemos llamado Community Web Caching, si está interesado en conocer como combinamos P2P, HTTP, comunidades virtuales, un servidor web y múltiples clientes para crear sitios web robustos, con pocos recursos, le recomendamos que continúe con esta lectura y cualquier comentario o recomendación estaremos sumamente complacidos  en atenderlo.

%----------------------------------------------------------------------------------------

\section{Comunidad Web Caché}

La idea del Community Web Caching nace a partir de las siguientes premisas:

\begin{enumerate}
\item Los usuarios cuentan con una gran cantidad de recursos en sus computadoras, los cuales son desperdiciados en más de un 80\%.
\item La mayoría del contenido en Internet es de interés común y no solo de la persona que lo publica, entonces si es de interés común porque solo unos cuantos deben mantenerlo. Porque no lo podemos mantener entre todos.
\item Quien publica contenido en Internet muy pocas veces tienen la capacidad para mantener grandes servidores que puedan  servir a grandes cantidades de usuarios.
\item Así como existen comunidades en Internet que se conforman por un interés común, por ejemplo hacer amigos, deportes, etc. Porque no crear una comunidad alrededor de un sitio web.
\marginpar{\myTitle \myVersion}
\end{enumerate}

La propuesta que se quiere implementar es dejar de lado el modelo tradicional de servir archivos mediante HTTP y dejar de lado también los modelos tradicionales de Web Caching, para utilizar las ventajas de ambos mas la creación de comunidades de caches y el uso de los recursos de los miembros de las comunidades caches y de esta manera utilizar algún protocolo, en este caso P2P para servir archivos desde diferentes localizaciones.

En este caso supongamos el siguiente ejemplo, contamos con el sitio www.cwc.org que sirve archivos de video de más de 30Mb, con un modelo tradicional HTTP, cliente-servidor, un cliente obtendría el archivo desde el servidor  imaginemos 1000 clientes haciendo esto, sería demasiado para el servidor si tiene recursos limitados, entonces tendríamos que pensar en aumentar los recursos del servidor, entre otras cosas. Nuestra propuesta es crear una comunidad alrededor de www.cwc.org supongamos que 30 de esos 1000 clientes forman parte de la comunidad y que tienen los recursos suficientes para servir archivos, entonces con esto podemos servir los archivos desde 31 lugares en lugar que desde uno y es evidente las ventajas que nos puede dar un modelo como este. Ahora que pasa si miembros de la comunidad no tienen suficientes recursos, bueno podemos hacer que en lugar de servir un archivo desde una sola localización se haga desde múltiples que queremos decir con esto, que en lugar de servir archivos grandes de 30MB o mas que sirvan solo una porción de estos, como en una comunidad en la cual cada miembro aporta lo que puede en beneficio de todos y para esto se utilizaría un protocolo como P2P, otra idea podría ser que los clientes con menos recursos sirvan archivos pequeños como por ejemplo imágenes. Esto le quitaría gran cantidad de trabajo al servidor ya que una imagen significa un thread más que se debe crear para servir un archivo.

La idea es que estas comunidades se conformen de manera automática y voluntaria, esto quiere decir que un usuario instala el plugin en su navegador web (Firefox) y el sitio al que trata de acceder puede usar el protocolo CWC entonces los contenidos accedidos quedan automáticamente en la cache y pueden ser servidos.

La cache podrá tener dos modos de operación, el primero sería que sea administrada por el usuario, esto quiere decir que cuando el usuario accede a un contenido este queda automáticamente en la cache y el usuario puede decidir si dejarlo o eliminarlo, el segundo seria manejado por el servidor esto quiere decir que cuando entramos en este modo el servidor tendrá derecho a utilizar los recursos asignados a la cache, entonces el servidor podrá enviar archivos que considere deberían de estar en la cache o borrar los que no considere necesarios.


%----------------------------------------------------------------------------------------